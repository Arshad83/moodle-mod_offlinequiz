\documentclass[12pt]{report}
\usepackage{german}

\setlength{\textwidth}{14.5cm}
\setlength{\textheight}{26.0cm}

\setlength{\topmargin}{-3.0cm}
%% \setlength{\headheight}{0.0cm}
%% \setlength{\oddsidemargin}{0.0cm} 
%% \setlength{\evensidemargin}{0.0cm}

\begin{document}

\subsection{Information on Offline Quiz Statistics}
\subsubsection{Definitions}
\begin{itemize}
\item $S$ is the set of students for which an offline quiz result exists. $|S|$ is the number of students in $S$.
\item $I_s$ is the set of items (questions) that have been answered by a student $s\in S$ in an offline quiz.
\item $S_i$ is the set of students that answered item (question) $i$.
\item $x_i(s)$ is the result that student $s$ achieved for item $i$.
\item $T_s = \sum\limits_{i\in I_s} x_i(s)$ is the complete result of student $s\in S$ adjusted to the maximum grade of the offline quiz.
\item $x_i(max)$ is the maximum grade for item $i$ adjusted to the maximum grade of the offline quiz. 
\item $x_i(min)$ is the minimum grade for item $i$ adjusted to the maximum grade of the offline quiz. 
\item $T_{max}=\sum\limits_{i\in I} x_i(max)$ is the maximum grade of the offline quiz.
\end{itemize}

\subsubsection{Statistiken - Overview}
We assume that a result exists for at least one student, i.e. $|S|>1$.

\paragraph{Average Grade}
We calculate the average grade of all students relative to the maximum grade of the offline quiz.
$\bar{T}=\frac{1}{|S|}\sum\limits_{s\in S}T_s$.

\paragraph{Median}
All grades in $T_s$ are sorted. If $|S|$ is odd the grade in the middle is displayed. If $|S|$ is even, the mean of the two middle grades is displayed.

\paragraph{Standard Deviation}
The standard deviation adjusted to the maximum grade of the offline quiz.

$SD = \sqrt{V(T)} = \sqrt{\frac{1}{|S| - 1}\sum\limits_{s\in S}(T_s - \bar{T})^2}$.

\paragraph{Skewness and Kurtosis}
Skewness is a measure of the asymmetry in a distribution. Kurtosis tells you if your distribution has more of a bulge, but thinner tails, or vice-versa. 

First, we calculate:

$m_2=\frac{1}{|S|}\sum\limits_{s\in S}{(T_s - \bar{T})^2}$

$m_3=\frac{1}{|S|}\sum\limits_{s\in S}{(T_s - \bar{T})^3}$

$m_4=\frac{1}{|S|}\sum\limits_{s\in S}{(T_s - \bar{T})^4}$\\

Then we calculate:\\

$k_2 = \frac{|S|}{|S| - 1}m_2 = V(T)$

$k_3 = \frac{|S|^2}{(|S|-1)(|S|-2)} m_3$

$k_4 = \frac{|S|^2}{(|S|-1)(|S|-2)(|S|-3)}\left((|S|+1)m_4-3(|S|-1)m_2^2\right)$\\

Then the skewness is defined as:\\

 $Skewness = \frac{k_3}{k_2^(3/2)}$\\

the kurtosis is defined as:\\

 $Kurtosis = \frac{k_4}{k_2^2}$


\paragraph{Coefficient of Internal Consistency (Cronbach's Alpha)}
The Coefficient of Internal Consistency (Cronbach's Alpha) is a measure for the internal consistency of a scale. It is commonly used as an estimate of the reliability of a psychometric test for a sample of examinees. It describes the  interrelatedness of the offline quiz questions. We calculate the Coefficient of Internal Consistency (CIC) as follows:

$CIC=\frac{|I|}{|I|-1}\left(1-\frac{1}{V(T)}\sum\limits_{i\in I}V(x_i)\right)$.


\subsubsection{Statistics - Question Analysis}
\paragraph{Notation}

$X_i(s) = T_s - x_i(s)$ is the remaining grade for student $s$ on item $i$.\\
$\bar{x}_i = \frac{1}{|S_i|}\sum\limits_{s\in S_i}x_i(s)$ is the average grade for item $i$ for all students. Analogously for other values.

\paragraph{Variance and Co-Variance}

We assume that at least two students have answered item $i$, i.e. $|S_i| > 1$.
The variance of the grade for item $i$ is defined as:

 $V(x_i) = \frac{1}{|S_i| - 1}\sum\limits_{s\in S_i} (x_i(s) - \bar{x}_i)^2$.

\noindent The co-variance of two quantities is defined as:\\
$C(x_i, X_i) = \frac{1}{|S_i| - 1}\sum\limits_{s\in S_i}(x_i(s) - \bar{x}_i)(X_i(s) - \bar{X}_i)$.


\paragraph{Facility Index}
The facility index is the average score of an item $i$ expressed as a percentage. If, for instance, two students have answered item $i$, where one student achieved $50\%$ and the other student achieved $100\%$, then the facility index is $75\%$.

$F_i = 100\cdot\frac{\bar{x}_i - x_i(min)}{x_i(max) - x_i(min)}$. 

\paragraph{Standard Deviation}
The standard deviation of a question shows how much variation or dispersion exists from the mean grade of that question. 

$SD_i = \frac{\sqrt{V(x_i)}}{x_i(max) - x_i(min)}$.

\paragraph{Discrimination Index}
This is the product moment correlation coefficient between $x_i$ and $X_i$, expressed on a percentage scale:

$D_i= \frac{C(x_i, X_i)}{\sqrt{V(x_i)V(X_i)}}$.

\ \\
\end{document}

%% 
%% End of file `statistics.tex'.
