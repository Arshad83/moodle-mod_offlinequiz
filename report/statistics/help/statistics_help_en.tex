\documentclass[12pt]{report}

\usepackage{html}

\setlength{\textwidth}{14.5cm}
\setlength{\textheight}{26.0cm}

\setlength{\topmargin}{-3.0cm}
%% \setlength{\headheight}{0.0cm}
%% \setlength{\oddsidemargin}{0.0cm} 
%% \setlength{\evensidemargin}{0.0cm}

\begin{document}

\subsection{Information on Offline Quiz Statistics}

\htmlref{Overview}{subsubsec:overview}\ \\
\htmlref{Question Analysis}{subsubsec:questions}\ \\
\htmlref{Definitions}{subsubsec:definitions}

%=======================================================
\subsection{Overview}
\label{subsubsec:overview}

We assume that a result exists for at least one student, i.e. $|S|>1$.

\paragraph{Total number of complete graded results}
Number of results taken into account for the statistics.

\paragraph{Maximum Grade Achievable}
The maximum grade that can be achieved by a student (best grade).

\paragraph{Highest/Lowest Grade Achieved}
The highest and the lowest grade achieved by a student.

\paragraph{Average Grade}
We calculate the average grade of all students relative to the maximum grade of the offline quiz.
$\bar{T}=\frac{1}{|S|}\sum\limits_{s\in S}T_s$.

\paragraph{Median}
All grades in $T_s$ are sorted. If $|S|$ is odd the grade in the middle is displayed. If $|S|$ is even, the mean of the two middle grades is displayed.


\paragraph{Standard Deviation}
The standard deviation adjusted to the maximum grade of the offline quiz.

$SD = \sqrt{V(T)} = \sqrt{\frac{1}{|S| - 1}\sum\limits_{s\in S}(T_s - \bar{T})^2}$.

\paragraph{Skewness and Kurtosis}
Skewness is a measure of the asymmetry in a distribution. Kurtosis tells you if your distribution has more of a bulge, but thinner tails, or vice-versa. 

First, we calculate:

$m_2=\frac{1}{|S|}\sum\limits_{s\in S}{(T_s - \bar{T})^2}$

$m_3=\frac{1}{|S|}\sum\limits_{s\in S}{(T_s - \bar{T})^3}$

$m_4=\frac{1}{|S|}\sum\limits_{s\in S}{(T_s - \bar{T})^4}$\\

Then we calculate:\\

$k_2 = \frac{|S|}{|S| - 1}m_2 = V(T)$

$k_3 = \frac{|S|^2}{(|S|-1)(|S|-2)} m_3$

$k_4 = \frac{|S|^2}{(|S|-1)(|S|-2)(|S|-3)}\left((|S|+1)m_4-3(|S|-1)m_2^2\right)$\\

Then the skewness is defined as:\\

 $Skewness = \frac{k_3}{k_2^(3/2)}$\\

the kurtosis is defined as:\\

 $Kurtosis = \frac{k_4}{k_2^2}$


\paragraph{Coefficient of Internal Consistency (Cronbach's Alpha)}
The Coefficient of Internal Consistency (Cronbach's Alpha) is a measure for the internal consistency of a scale. It is commonly used as an estimate of the reliability of a psychometric test for a sample of examinees. It describes the  interrelatedness of the offline quiz questions. We calculate the Coefficient of Internal Consistency (CIC) as follows:

$CIC=\frac{|I|}{|I|-1}\left(1-\frac{1}{V(T)}\sum\limits_{i\in I}V(x_i)\right)$.


\paragraph{Error Ratio and Standard Error}

These values have to do with estimating how reliable the test scores are.
If you take the view that the score the student got on the test on the day is a combination
 of their actual ability and a random error
  (how lucky they were on the day of the test),
   the the standard error is an estimate of the luck factor. So if SE ~= 10, and the student scored 60, then you can be quite confident
    that their actual ability is between 50 and 70.\\

The error ratio is calculated as:

$ER=100*\sqrt{1-\frac{CIC}{100}}$.

The standard error is calculated as:

$SE=\frac{ER}{100}SD$

\ \\

\subsection{Question Analysis}
\label{subsubsec:questions}

\paragraph{Facility Index}
The facility index is the average score of an item $i$ expressed as a percentage. If, for instance, two students have answered item $i$, where one student achieved $50\%$ and the other student achieved $100\%$, then the facility index is $75\%$.

$F_i = 100\cdot\frac{\bar{x}_i - x_i(min)}{x_i(max) - x_i(min)}$. 

\paragraph{Standard Deviation}
The standard deviation of a question shows how much variation or dispersion exists from the mean grade of that question. 

$SD_i = \frac{\sqrt{V(x_i)}}{x_i(max) - x_i(min)}$.

\paragraph{Intended Question Weight}

How much this question was supposed to contribute to determining the overall test score.

$IQW_p = 100\frac{x_p(max) - x_p(min)}{T_{max} - T_{min}}$.\\

\paragraph{Effective Question Weight}

This is an estimate of what proportion of the variance in the students' test scores is due this question.

$EQW_p = 100\frac{\sqrt{C(x_p, T)}}{\sum_{p \in P}\sqrt{C(x_p, T)}}.$\\

 
\paragraph{Discrimination Index}
This is the product moment correlation coefficient between $x_i$ and $X_i$, expressed on a percentage scale:

$D_i= \frac{C(x_i, X_i)}{\sqrt{V(x_i)V(X_i)}}$.


\subsection{Definitions}
\label{subsubsec:definitions}
\begin{itemize}
\item $S$ is the set of students for which an offline quiz result exists. $|S|$ is the number of students in $S$.
\item $I_s$ is the set of items (questions) that have been answered by a student $s\in S$ in an offline quiz.
\item $S_i$ is the set of students that answered item (question) $i$.
\item $x_i(s)$ is the result that student $s$ achieved for item $i$.
\item $T_s = \sum\limits_{i\in I_s} x_i(s)$ is the complete result of student $s\in S$ adjusted to the maximum grade of the offline quiz.
\item $x_i(max)$ is the maximum grade for item $i$ adjusted to the maximum grade of the offline quiz. 
\item $x_i(min)$ is the minimum grade for item $i$ adjusted to the maximum grade of the offline quiz. 
\item $T_{max}=\sum\limits_{i\in I} x_i(max)$ is the maximum grade of the offline quiz.
\item $X_i(s) = T_s - x_i(s)$ is the remaining grade for student $s$ on item $i$.\\
\item $\bar{x}_i = \frac{1}{|S_i|}\sum\limits_{s\in S_i}x_i(s)$ is the average grade for item $i$ for all students. Analogously for other values.
\end{itemize}

We assume that at least two students have answered item $i$, i.e. $|S_i| > 1$.
Then we define the following values:
\begin{itemize}
\item $V(x_i) = \frac{1}{|S_i| - 1}\sum\limits_{s\in S_i} (x_i(s) - \bar{x}_i)^2$ is the variance of the grade for item $i$.
\item $C(x_i, X_i) = \frac{1}{|S_i| - 1}\sum\limits_{s\in S_i}(x_i(s) - \bar{x}_i)(X_i(s) - \bar{X}_i)$ is the co-variance of $x_i$ and $X_i$.
\end{document}

%% 
%% End of file `statistics_help_en.tex'.
